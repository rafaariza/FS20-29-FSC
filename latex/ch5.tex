\chapter{Conclusiones}
\begin{itemize}
    \item El uso del lenguaje Python para simulaciones cosmológicas es totalmente esencial. Por motivos como que, de la comunidad que hay tras él podemos tomar funciones ya creadas que son necesarias en nuestros algoritmos.
    \item La transformada rápida de Fourier y los números aleatorios han sido realmente relevantes en el desarrollo del trabajo. El modelizado por ordenador de los campos gaussianos aleatorios no hubiera sido posible sin estos dos materiales, ya que se usaron \textit{arrays} complejos de tamaño hasta \(512^3\) y debían ser lo suficientemente aleatorios para dar el modelo como válido.
    \item Se ha podido comprobar que los campos de densidad primordiales, generados con el espectro de potencias \(\symrm{P}_0(k)\), muestran dominancia de fluctuaciones grandes para \(n_s<0\) y de fluctuaciones pequeñas para \(n_s>0\). El caso Harrison-Zel'dovich \((n_s=1)\) ha devuelto el esperado campo de fluctuaciones homogéneo en grandes escalas pero con fluctuaciones notables en escalas más pequeñas (esto se comprobó amplificando la imagen de dicha realización).
    \item La incorporación de la función de transferencia ha permitido reproducir la distribución espacial de las fluctuaciones en el CMB.
    \item Los cortes bidimensionales de las realizaciones han aportado lo necesario para entender los espectros de potencias. Gracias a la animación tridimensional se ha verificado que existe correlación espacial en los campos gaussianos aleatorios.
\end{itemize}
\chapter*{Conclusions}
\begin{itemize}
    \item The use of the Python language for cosmological simulations is absolutely essential. For reasons such as, from the community behind it we can take existing functions that are needed in our algorithms.
    \item The fast Fourier transform and random numbers have been really relevant in the development of the work. The computer modeling of the Gaussian random fields would not have been possible without these two materials, since complex arrays of size up to \(512^3\) were used and had to be random enough to give the model as valid.
    \item It has been found that the primordial density fields, generated with the power spectrum \(\symrm{P}_0(k)\), show dominance on large fluctuations for \(n_s<0\) and on small fluctuations for \(n_s>0\). The Harrison-Zel'dovich case \((n_s=1)\) has returned the expected homogeneous fluctuation field at large scales but with noticeable fluctuations at smaller scales (this was verified by amplifying the image of such a realization).
    \item The incorporation of the transfer function has made it possible to reproduce the spatial distribution of fluctuations in the CMB.
    \item Two-dimensional slices of the realizations have provided what is needed to understand the power spectra. Thanks to the three-dimensional animation, spatial correlation in the Gaussian random fields has been verified.
\end{itemize}