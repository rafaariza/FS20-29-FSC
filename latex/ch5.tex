\chapter{Conclusiones}
\begin{itemize}
    \item El lenguaje Python es una herramienta indispensable en cosmología por dos motivos: alta capacidad en cálculos con \textit{arrays} gracias a librerías como Numpy y facilidad para construir algoritmos al poder utilizar librerías de otros usuarios que contengan funciones que son necesarias en nuestro código. Por estas dos razones ha sido posible confeccionar un algoritmo que sea acorde a las ecuaciones teóricas y simule los campos gaussianos aleatorios de densidad.
    \item La transformada rápida de Fourier es uno de los algoritmos más ingeniosos de la historia, haciendo que el cálculo de transformadas discretas de Fourier sea eficiente cuando el número de datos de entrada es grande. La generación de números aleatorios del módulo \texttt{numpy.random} ha sido realmente relevante en el desarrollo del trabajo por el hecho de ser buenos números aleatorios y aportar el carácter estocástico necesario. El modelizado por ordenador de los campos gaussianos aleatorios no hubiera sido posible sin estos dos materiales, ya que se usaron \textit{arrays} complejos de tamaño hasta \(512^3\) y debían ser lo suficientemente aleatorios para dar el modelo como válido.
    \item Ambos modelos, espectro de potencias inicial y en época de recombinación, han sido sometidos a diferentes simulaciones derivando en resultados que se corresponden con la teoría y con algunas observaciones experimentales como el CMB.
    \item Los cortes bidimensionales de las realizaciones han aportado lo necesario para entender los espectros de potencias, ya que solo con intuición y ecuaciones es un concepto más abstracto. Gracias a la animación tridimensional se ha verificado que existe correlación espacial en los campos gaussianos aleatorios.
\end{itemize}
\chapter*{Conclusions}
\begin{itemize}
    \item The Python language is an indispensable tool in cosmology for two reasons: high capacity in calculations with arrays thanks to libraries such as Numpy and ease of building algorithms by being able to use other users' libraries that contain functions that are necessary in our code. For these two reasons it has been possible to build an algorithm that is in line with the theoretical equations and simulates the Gaussian random density fields.
    \item The fast Fourier transform is one of the most ingenious algorithms in history, making the computation of discrete Fourier transforms efficient when the number of input data is large. The random number generation of the \texttt{numpy.random} module has been really relevant in the development of this work because they are good random numbers and provide the necessary stochastic character. The computer modeling of the Gaussian random fields would not have been possible without these two tools, since complex arrays of size up to \(512^3\) were used and had to be random enough to give the model as valid.
    \item Both models, initial power spectrum and at recombination epoch, have been subjected to different simulations deriving results that correspond with theory and with some experimental observations such as the CMB.
    \item The two-dimensional slices of the realizations have provided what is necessary to understand the power spectra, since only with intuition and equations it is a more abstract concept. Thanks to the three-dimensional animation, it has been verified that there is spatial correlation in Gaussian random fields.
\end{itemize}