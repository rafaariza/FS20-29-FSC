\chapter{Materiales y métodos}
En este capítulo se describe cómo fueron realizadas las simulaciones de los campos gaussianos aleatorios para poder ser replicadas. También se especifican los materiales usados en todo el proceso.

Se generarán realizaciones tridimensionales (de las que se tomarán cortes bidimensionales), así como una visualización animada de la tercera dimensión en formato GIF o MP4, del campo de fluctuaciones gaussiano para ambos espectros de potencias y diferentes valores de \(n_s\). En primer lugar para el espectro de potencias primordial \(\symrm{P}_0(k)\):
\begin{itemize}
    \item \(n_s=0\). Ruido blanco, el fondo sobre el que perturbamos. Servirá para comprobar el funcionamiento.
    \item \(n_s=1\). Harrison-Zel'dovich. Grandes escalas, homogéneo predominando las fluctuaciones pequeñas.
    \item \(n_s=-3\). Pequeñas escalas del universo, se aprecian fluctuaciones más grandes.
\end{itemize}
Y al final, para la simulación más realista usando~\ref{eq::espectrolineal} y~\ref{eq::transff}:
\begin{itemize}
    \item \(n_s=5\). Dará lugar a \(P(k)\propto k\) que corresponde a grandes escalas y ha de ser homogéneo con pequeñas fluctuaciones.
    \item \(n_s=1\). Conducirá a \(P(k)\propto k^{-3}\) que apuntará a escalas más pequeñas donde las fluctuaciones son más grandes.
\end{itemize}

