\chapter*{Resumen}
\addcontentsline{toc}{chapter}{Resumen. Palabras clave}
Para poder explicar ciertas peculiaridades que aparecen en la teoría del \textit{Big Bang} se recurre al concepto de inflación. Al tratar esta idea de manera cuántica, aparece un mecanismo productor de las fluctuaciones de densidad primordiales que dieron lugar a las complejas estructuras del universo como galaxias, estrellas y la vida misma.

En este trabajo se da una introducción del marco teórico del \textit{Big Bang} y sus problemas e inflación, así como de las fluctuaciones que hubo en el universo primordial y la estadística necesaria para ser comprendidas. Se va a estudiar que la distribución inicial de dichas fluctuaciones puede ser modelada mediante un campo de densidad aleatorio con estadística gaussiana (el cual está caracterizado completamente por su espectro de potencias primordial \(\symrm{P}_0(k)\propto k^{n_s}\)), tal y como predicen las teorías inflacionarias. Para poder profundizar, más allá de la intuición, en el significado de dicha estadística gaussiana se ha programado un algoritmo en el lenguaje Python que es capaz de simular numéricamente estas fluctuaciones en el universo primordial y, mediante la incorporación de la función de transferencia \(T(k)\) al espectro de potencias \(\symrm{P}(k)\propto k^{n_s}T^2(k)\), en la época de recombinación. Esto se hace mediante el uso de herramientas como la transformada rápida de Fourier o los números aleatorios, de las cuales también se aporta una introducción teórica.

Finalmente, se han representado las simulaciones numéricas para diferentes valores de \(n_s\) mediante mapas de calor, los cuales son los resultados cenitales del trabajo.
\paragraph{Palabras clave:} cosmología; computacional; inflación; estructura a gran escala
\chapter*{Abstract}
\addcontentsline{toc}{chapter}{Abstract. Keywords}
In order to explain certain peculiarities that appear in the Big Bang theory, we resort to the concept of inflation. By treating this idea in a quantum way, a mechanism that produced the primordial density fluctuations that gave rise to the complex structures of the universe such as galaxies, stars and life itself appears.

This paper gives an introduction to the theoretical framework of the Big Bang and its problems and inflation, as well as the fluctuations that occurred in the primordial universe and the statistics needed to understand them. The initial distribution of these fluctuations can be shaped by means of a random density field with Gaussian statistics (which is fully characterized by its primordial power spectrum \(\symrm{P}_0(k)\propto k^{n_s}\)), as predicted by inflationary theories. For the purpose to delve, beyond intuition, into the meaning of such a Gaussian statistics, an algorithm has been programmed in the Python language that is able to numerically simulate these fluctuations in the primordial universe and in the recombination epoch by incorporating the transfer function \(T(k)\) into the power spectrum \(\symrm{P}(k)\propto k^{n_s}T^2(k)\). This is done by using tools such as the fast Fourier transform or random numbers, of which a theoretical introduction is also provided.

Finally, the numerical simulations for different \(n_s\) values are represented by means of heat maps, which are the peak results of the work.
\paragraph{Keywords:} cosmology; computational; inflation; large scale structure