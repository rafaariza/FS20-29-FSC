\chapter*{Resumen}
\addcontentsline{toc}{chapter}{Resumen. Palabras clave}
Para poder explicar ciertas peculiaridades que aparecen en la teoría del \textit{Big Bang} se recurrió al concepto de inflación. Al tratar esta idea de manera cuántica, apareció un mecanismo productor de las fluctuaciones de densidad primordiales que dieron lugar a las complejas estructuras del universo como galaxias, estrellas y la vida misma.

En este trabajo se va a dar una introducción del marco teórico del \textit{Big Bang} y sus problemas e inflación, así como de las fluctuaciones que hubo en el universo temprano y la estadística necesaria para ser comprendidas. Para poder profundizar, más allá de la intuición, en el significado de dicha estadística se va a programar un algoritmo en el lenguaje Python que sea capaz de simular numéricamente las fluctuaciones en el universo primordial y en la época de recombinación. Esto se hará mediante el uso de herramientas como la transformada rápida de Fourier o los números aleatorios, de las cuales también se aportará una introducción teórica.

Finalmente, se representarán las simulaciones numéricas con diferentes mapas de calor que serán los resultados cenitales del trabajo.
\paragraph{Palabras clave:} cosmología; computacional; inflación; estructura a gran escala
\chapter*{Abstract}
\addcontentsline{toc}{chapter}{Abstract. Keywords}
In order to explain certain peculiarities that appear in the Big Bang theory, the concept of inflation was relied upon. By treating this idea in a quantum way, a mechanism that produced the primordial density fluctuations that gave rise to the complex structures of the universe such as galaxies, stars and life itself appeared.

In this paper we will give an introduction of the theoretical framework of the Big Bang and its problems and inflation, as well of the fluctuations in the early universe and the statistics needed to understand them. In order to go deeper, beyond intuition, in the meaning of such statistics, an algorithm in the Python language that is able to numerically simulate the fluctuations in the primordial universe and in the recombination epoch will be programmed. This will be done through the use of tools such as the fast Fourier transform or the random numbers of which a theoretical introduction will also be given.

Finally, the numerical simulations will be represented with different heat maps which will be the peak results of the work.
\paragraph{Keywords:} cosmology; computational; inflation; large scale structure