\section{Introducción}
\begin{frame}[noframenumbering]{Introducción}
    \only<1>{
        \begin{block}{Inflación}
            El Universo no es como las ecuaciones del \textit{Hot Big Bang} orquestan. Aparecen una serie de problemas teóricos, entre ellos: \EMPH{el problema del horizonte}
        \end{block}
    }
    \only<2>{
        \begin{figure}
            \centering
            \def\svgwidth{0.8\textwidth}
            \input{../svg/horizonproblemslides.pdf_tex}
        \end{figure}
    }
    \only<3>{
        \begin{block}{Inflación}
            El Universo no es como las ecuaciones del \textit{Hot Big Bang} orquestan. Aparecen una serie de problemas teóricos, entre ellos: \EMPH{el problema del horizonte} y el \EMPH{problema de la planitud}:
            \begin{equation*}
                H^2=\frac{8\pi G}{3}\rho-\frac{kc^2}{a^2},\qquad \rho_c\equiv\frac{3H^2}{8\pi G}=\rho.
            \end{equation*}
            Inflación es capaz de \EMPH{resolver} estos dos problemas y además provee de manera natural un \EMPH{mecanismo generador} de las fluctuaciones primordiales.
        \end{block}
    }
    \only<4>{
        \begin{figure}
            \centering
            \def\svgwidth{0.7\textwidth}
            \input{../svg/horizonproblemsolslides.pdf_tex}
        \end{figure}}
    \only<5>{
        \begin{block}{Inflación}
            \begin{itemize}[topsep=0pt]
                \okitem Marco teórico \(\longrightarrow\) Campos escalares.
                \okitem El campo escalar es el \EMPH{inflatón} \(\phi(t,\symbf{x})\).
                \okitem Su ec. de estado \(\symscr{w}={\symcal{P_\phi}}/{\rho_\phi}<-1/3\).
                \okitem Tiene una densidad de e. potencial \(V(\phi)\) y de e. cinética \(\dot{\phi}^2/2\).
                \okitem Su ec. de movimiento \(\ddot{\phi}+3H\dot{\phi}+V'(\phi)=0\).
                \okitem Escenario \textit{slow-roll}.
            \end{itemize}
        \end{block}
    }
    \only<6>{
        \begin{figure}
            \centering
            \def\svgwidth{0.65\textwidth}
            \input{../svg/fluctuations.pdf_tex}
        \end{figure}
    }
\end{frame}
\begin{frame}{Introducción}
    \begin{block}{Espectro de potencias y función de transferencia}
        \begin{center}
            Fluctuaciones cuánticas \(\delta\phi(\symbf{x})\) \(\longrightarrow\) Fluctuaciones clásicas densidad de energía \(\delta\rho(t,\symbf{x})\)
        \end{center}
        Estas fluctuaciones clásicas son descritas en el \EMPH{régimen lineal}:
        \begin{equation*}
            \delta(t,\symbf{x})=\frac{\delta\rho(t,\symbf{x})}{\overbar{\rho}(t)}\ll 1.
        \end{equation*}
        El espacio de Fourier entra en juego para facilitar el estudio de las fluctuaciones (``modos'').\\
        En cosmología nos importa el comportamiento estadístico de las cantidades estudiadas. El campo de fluctuaciones \(\delta\) se modela con un \EMPH{campo aleatorio}.
    \end{block}
\end{frame}
\begin{frame}{Introducción}
    \begin{block}{Espectro de potencias y función de transferencia}
        Estamos interesados en la \EMPH{correlación espacial} de las fluctuaciones:
        \begin{equation*}
            \xi\left(\symbf{x},\symbf{x}',t\right)\equiv\left\langle\delta(\symbf{x},t),\delta(\symbf{x}',t)\right\rangle=\int\symcal{D}\delta\,\symbb{P}\left[\delta\right]\,\delta(\symbf{x},t)\delta(\symbf{x}',t).
        \end{equation*}
        \begin{center}
            Homogeneidad e isotropía \(\implies\xi\left(\symbf{x},\symbf{x}',t\right)=\xi(r)\).
        \end{center}
        En el espacio de Fourier:
        \begin{equation*}
            \left\langle\delta(\symbf{k})\delta^{*}(\symbf{k}')\right\rangle\equiv (2\pi)^3\,\delta_{\symrm{D}}(\symbf{k}-\symbf{k}')\,\symrm{P}(k),
        \end{equation*}
        donde \(\symrm{P}(k)\) es el espectro de potencias.\\[.3cm]
        En el caso que veremos adelante, \(\symrm{P}(k)\) lleva toda la información del campo aleatorio. La evolución gravitatoria de las fluctuaciones es descrita con la \EMPH{función de transferencia} \(T(k)\).
    \end{block}
\end{frame}
\begin{frame}{Introducción}
    \begin{block}{Campos aleatorios gaussianos}
        En este tipo de campos aleatorios, la FDP es una gaussiana en cada punto del espacio. Existe \EMPH{evidencia observacional} suficiente para tomar este campo como modelo del campo de densidad primordial.
        \begin{center}
            El \EMPH{espectro de potencias} define completamente un campo aleatorio gaussiano.
        \end{center}
        Para los diferentes vectores de onda \(\symbf{k}\):
        \begin{equation*}
            \delta(\symbf{k})=A_{\symbf{k}}+\symrm{i}B_{\symbf{k}}=\left|\delta(\symbf{k})\right|\,\symrm{e}^{\symrm{i}2\pi\phi_{\symbf{k}}},\quad \left|\delta(\symbf{k})\right|\sim\mathcal{N}(0,\symrm{P}(k))\ \wedge\ \phi_{\symbf{k}}\sim U(0,1).
        \end{equation*}
    \end{block}
\end{frame}