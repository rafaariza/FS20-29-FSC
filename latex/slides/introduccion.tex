\section{Introducción}
\begin{frame}{Introducción}
    \only<1>{
    \begin{block}{Inflación}
        En el \textit{Hot Big Bang} aparecen una serie de problemas teóricos, entre ellos: \EMPH{el problema del horizonte}
    \end{block}}
    \only<2>{
    \begin{figure}
        \centering
        \def\svgwidth{0.7\textwidth}
        \input{../svg/horizonproblemslides.pdf_tex}
    \end{figure}}
    \only<3>{
    \begin{block}{Inflación}
    En el \textit{Hot Big Bang} aparecen una serie de problemas teóricos, entre ellos: \EMPH{el problema del horizonte} y el \EMPH{problema de la planitud}:
    \begin{equation*}
        H^2=\frac{8\pi G}{3}\rho-\frac{kc^2}{a^2},\qquad \rho_c\equiv\frac{3H^2}{8\pi G}=\rho.
    \end{equation*}
    Inflación es capaz de resolver estos dos problemas y además provee de manera natural un mecanismo generador de las fluctuaciones primordiales.
    \end{block}}
    \only<4>{
    \begin{figure}
        \centering
        \def\svgwidth{0.7\textwidth}
        \input{../svg/horizonproblemsolslides.pdf_tex}
    \end{figure}}
    \only<5>{
    \begin{figure}
        \centering
        \def\svgwidth{0.65\textwidth}
        \input{../svg/fluctuations.pdf_tex}
    \end{figure}}
\end{frame}
\begin{frame}{Introducción}
    \begin{block}{Espectro de potencias y función de transferencia}
        Pue' eso.
    \end{block}
\end{frame}