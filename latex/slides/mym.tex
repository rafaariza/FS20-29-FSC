\section{Materiales y métodos}
\begin{frame}[noframenumbering]{Materiales y métodos}
    \begin{block}{FFT y números aleatorios gaussianos}
        Al simular por ordenador la transformada de Fourier se convierte en una DFT.
        \begin{center}
            \begin{minipage}{.5\linewidth}
                \begin{itemize}
                    \item[\textcolor{ngmark}{\ngmark}] DFT (definición directa) \(\longrightarrow O(N^2)\)
                    \item[\textcolor{okmark}{\okmark}] FFT (algoritmo ingenioso) \(\longrightarrow O(N\log_2N)\)
                \end{itemize}
            \end{minipage}
        \end{center}
        \pause
        \vspace{.3cm}
        El \EMPH{campo aleatorio gaussiano} es computado mediante el uso de número aleatorios bajo una estadística normal (o gaussiana) con media nula y desviación típica \(\sqrt{\symrm{P}(k)}\).
    \end{block}
\end{frame}
\begin{frame}[fragile]{Materiales y métodos}
    \begin{block}{Método}
        \centering
        \begin{tikzpicture}[node distance=1.7cm]
            \node (uno) [block] {Caja \((L,N)\)};
            \node (dos) [block, right of=uno, xshift=5cm] {Determinamos \(k_j=\frac{2\pi}{L}j,\quad j\in\left(-\frac{N}{2},\cdots,\frac{N}{2}\right)\)};
            \node (tres) [block, below of=dos] {Creamos el array \(k_j=\sqrt{\sum_dk^2_{d,j}}\ \) y el \(G(0,1)\)};
            \node (cinco) [block, below of=tres] {Calculamos el array \(\symrm{P}(k_j)\)};
            \node (seis) [block, below of=cinco] {Computamos el array \(\delta(\symbf{k})=G\sqrt{\symrm{P}(k_j)}\)};
            \node (siete) [block, left of=seis, xshift=-5cm, text width=4cm] {Campo de fluctuaciones real \(\delta(\symbf{x})=L^3\symcal{F}^{-1}\left[\delta(\symbf{k})\right]\)};
            \draw [arrow] (uno) -- (dos);
            \draw [arrow] (dos) -- (tres);
            \draw [arrow] (tres) -- (cinco);
            \draw [arrow] (cinco) -- (seis);
            \draw [arrow] (seis) -- (siete);
        \end{tikzpicture}
    \end{block}
\end{frame}