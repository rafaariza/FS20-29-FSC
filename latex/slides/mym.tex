\section{Materiales y métodos}
\begin{frame}[noframenumbering]{Materiales y métodos}
    \begin{block}{FFT y números aleatorios gaussianos}
        Al simular por ordenador la transformada de Fourier se convierte en una DFT.
        \begin{center}
            \begin{minipage}{.5\linewidth}
                \begin{itemize}
                    \item[\textcolor{ngmark}{\ngmark}] DFT (definición directa) \(\longrightarrow O(N^2)\)
                    \item[\textcolor{okmark}{\okmark}] FFT (algoritmo ingenioso) \(\longrightarrow O(N\log_2N)\)
                \end{itemize}
            \end{minipage}
        \end{center}
        \pause
        \vspace{.3cm}
        El campo aleatorio gaussiano es computado mediante el uso de número aleatorios bajo una estadística normal (o gaussiana) con media nula y desviación típica \(\symrm{P}(k)\).
    \end{block}
\end{frame}
\begin{frame}[fragile]{Materiales y métodos}
    \only<1>{
        \begin{block}{Método (librerías)}
            \vspace*{-1em}
            \lstinputlisting[linerange=1-18, language = Python]{../../scripts64/FS2029FSC.py}
        \end{block}
    }
    \only<2>{
        \begin{block}{Método (inicialización de la clase)}
            \vspace*{-1em}
            \lstinputlisting[linerange=23-42, language = Python]{../../scripts64/FS2029FSC.py}
        \end{block}
    }
    \only<3>{
        \begin{block}{Método (generador)}
            \vspace*{-1em}
            \lstinputlisting[linerange=44-64, language = Python]{../../scripts64/FS2029FSC.py}
        \end{block}
    }
    \only<4>{
        \begin{block}{Método (generador)}
            \lstinputlisting[linerange=66-76, language = Python]{../../scripts64/FS2029FSC.py}
        \end{block}
    }
    \only<5>{
        \begin{block}{Método (espectro primordial)}
            \lstinputlisting[linerange=121-126, language = Python]{../../scripts64/FS2029FSC.py}
        \end{block}
    }
    \only<6>{
        \begin{block}{Método (espectro en época de recombinación)}
            \lstinputlisting[linerange=110-119, language = Python]{../../scripts64/FS2029FSC.py}
        \end{block}
    }
\end{frame}