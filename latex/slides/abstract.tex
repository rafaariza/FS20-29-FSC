\section{Abstract}
\begin{frame}[noframenumbering]{Abstract}
    \footnotesize In order to explain certain peculiarities that appear in the Big Bang theory, we resort to the concept of inflation. By treating this idea in a quantum way, a mechanism that produced the primordial density fluctuations that gave rise to the complex structures of the universe such as galaxies, stars and life itself appears.

    This paper gives an introduction to the theoretical framework of the Big Bang and its problems and inflation, as well as the fluctuations that occurred in the primordial universe and the statistics needed to understand them. The initial distribution of these fluctuations can be shaped by means of a random density field with Gaussian statistics (which is fully characterized by its primordial power spectrum \(\symrm{P}_0(k)\propto k^{n_s}\)), as predicted by inflationary theories. For the purpose to delve, beyond intuition, into the meaning of such a Gaussian statistics, an algorithm has been programmed in the Python language that is able to numerically simulate these fluctuations in the primordial universe and in the recombination epoch by incorporating the transfer function \(T(k)\) into the power spectrum \(\symrm{P}(k)\propto k^{n_s}T^2(k)\). This is done by using tools such as the fast Fourier transform or random numbers, of which a theoretical introduction is also provided.
    
    Finally, the numerical simulations for different \(n_s\) values are represented by means of heat maps, which are the peak results of the work.
\end{frame}