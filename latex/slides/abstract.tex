\section{Abstract}
\begin{frame}[noframenumbering]{Abstract}
    In order to explain certain \EMPH{peculiarities} that appear in the Big Bang theory, we resort to the concept of \EMPH{inflation}.

    This work gives an introduction to the theoretical framework of the \EMPH{Big Bang and its problems} and inflation, as well as the \EMPH{fluctuations} that occurred in the primordial universe and the statistics needed to understand them. The initial distribution of these fluctuations can be shaped by means of a random density field with \EMPH{Gaussian statistics}. For the purpose to delve into the meaning of such a Gaussian statistics, an \EMPH{algorithm} has been programmed in the \EMPH{Python} language. This is done by using tools such as the \EMPH{fast Fourier transform} or random numbers.
    
    Finally, the numerical simulations are represented by means of heat maps, which are the peak results of the work.
\end{frame}